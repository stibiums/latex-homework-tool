\documentclass[12pt,a4paper]{article}

% ========================== 包导入 ==========================
\usepackage[UTF8]{ctex}                    % 中文支持
\usepackage[margin=2.5cm]{geometry}        % 页面设置
\usepackage{amsmath,amssymb,amsthm}        % 数学公式
\usepackage{graphicx}                      % 图片插入
\usepackage{float}                         % 图片位置控制
\usepackage{subcaption}                    % 子图支持
\usepackage{booktabs}                      % 表格美化
\usepackage{array}                         % 表格增强
\usepackage{enumitem}                      % 列表控制
\usepackage{xcolor}                        % 颜色支持
\usepackage{listings}                      % 代码块
\usepackage{fancyhdr}                      % 页眉页脚
\usepackage{hyperref}                      % 超链接
\usepackage{tikz}                          % 绘图
\usepackage{pgfplots}                      % 数据图表
\usepackage{algorithm}                     % 算法环境
\usepackage{algorithmic}                   % 算法伪代码
\usepackage{bm}                           % 粗体数学符号
\usepackage{url}                          % URL支持
\usepackage{tabularx}                     % 自适应表格
\usepackage{multirow}                     % 表格多行合并
\usepackage{footnote}                     % 脚注增强

% ========================== 可配置参数 ==========================
% 在这里修改课程信息
\newcommand{\coursename}{课程名称}         % 修改为你的课程名称
\newcommand{\hwnum}{1}                     % 修改为作业编号
\newcommand{\studentname}{姓名}            % 修改为你的姓名
\newcommand{\studentid}{学号}              % 修改为你的学号
\newcommand{\department}{院系}             % 修改为你的院系
\newcommand{\semester}{2025年秋季学期}     % 修改为当前学期

% ========================== 基础设置 ==========================
\pagestyle{fancy}
\fancyhf{}
\fancyhead[L]{\coursename 作业\hwnum}
\fancyhead[R]{\thepage}
\renewcommand{\headrulewidth}{0.4pt}

% 行距设置
\linespread{1.2}

% 段落间距
\setlength{\parskip}{0.5em}

% ========================== 代码设置 ==========================
\lstdefinestyle{pythonstyle}{
    language=Python,
    basicstyle=\ttfamily\small,
    keywordstyle=\color{blue}\bfseries,
    commentstyle=\color{green!60!black},
    stringstyle=\color{red},
    numbers=left,
    numberstyle=\tiny\color{gray},
    stepnumber=1,
    numbersep=8pt,
    backgroundcolor=\color{gray!10},
    showspaces=false,
    showstringspaces=false,
    showtabs=false,
    frame=single,
    rulecolor=\color{black!30},
    tabsize=4,
    captionpos=b,
    breaklines=true,
    breakatwhitespace=false,
    escapeinside={\%*}{*)}
}

\lstdefinestyle{cppstyle}{
    language=C++,
    basicstyle=\ttfamily\small,
    keywordstyle=\color{blue}\bfseries,
    commentstyle=\color{green!60!black},
    stringstyle=\color{red},
    numbers=left,
    numberstyle=\tiny\color{gray},
    stepnumber=1,
    numbersep=8pt,
    backgroundcolor=\color{gray!10},
    showspaces=false,
    showstringspaces=false,
    showtabs=false,
    frame=single,
    rulecolor=\color{black!30},
    tabsize=4,
    captionpos=b,
    breaklines=true,
    breakatwhitespace=false,
}

\lstdefinestyle{matlabstyle}{
    language=Matlab,
    basicstyle=\ttfamily\small,
    keywordstyle=\color{blue}\bfseries,
    commentstyle=\color{green!60!black},
    stringstyle=\color{red},
    numbers=left,
    numberstyle=\tiny\color{gray},
    stepnumber=1,
    numbersep=8pt,
    backgroundcolor=\color{gray!10},
    showspaces=false,
    showstringspaces=false,
    showtabs=false,
    frame=single,
    rulecolor=\color{black!30},
    tabsize=4,
    captionpos=b,
    breaklines=true,
    breakatwhitespace=false,
}

\lstset{style=pythonstyle}

% ========================== 数学设置 ==========================
\numberwithin{equation}{section}
\newtheorem{theorem}{定理}[section]
\newtheorem{lemma}[theorem]{引理}
\newtheorem{proposition}[theorem]{命题}
\newtheorem{corollary}[theorem]{推论}
\newtheorem{definition}{定义}[section]
\newtheorem{example}{例}[section]
\newtheorem{remark}{注}[section]

% ========================== 自定义命令 ==========================
% 向量表示
\newcommand{\vect}[1]{\boldsymbol{#1}}
% 矩阵表示
\newcommand{\mat}[1]{\boldsymbol{#1}}
% 答案框
\newcommand{\answerbox}[1]{\fbox{\parbox{\textwidth}{#1}}}
% 解答开始
\newcommand{\solution}{\textbf{解:}}
% 重点标记
\newcommand{\important}[1]{\textcolor{red}{\textbf{#1}}}
% 注意事项
\newcommand{\note}[1]{\textcolor{orange}{\textbf{注:}#1}}
% 提示
\newcommand{\hint}[1]{\textcolor{blue}{\textbf{提示:}#1}}

% ========================== 超链接设置 ==========================
\hypersetup{
    colorlinks=true,
    linkcolor=blue,
    filecolor=magenta,
    urlcolor=cyan,
    citecolor=green,
    pdftitle={\coursename 作业\hwnum},
    pdfauthor={\studentname},
    pdfsubject={\coursename},
}

% ========================== 开始文档 ==========================
\begin{document}

% ========================== 标题页 ==========================
\begin{titlepage}
    \centering
    \vspace*{2cm}

    {\LARGE \textbf{北京大学}}\\[1cm]
    {\Large \coursename}\\[1.5cm]

    \rule{\linewidth}{0.5mm}\\[0.4cm]
    {\huge \textbf{作业 \hwnum}}\\[0.4cm]
    \rule{\linewidth}{0.5mm}\\[1.5cm]

    \begin{minipage}{0.6\textwidth}
        \begin{flushleft}
            \large
            \textbf{姓名:} \studentname\\
            \textbf{学号:} \studentid\\
            \textbf{院系:} \department\\
            \textbf{课程:} \coursename\\
            \textbf{学期:} \semester
        \end{flushleft}
    \end{minipage}

    \vfill

    {\large 提交日期:\today}

\end{titlepage}

% ========================== 目录(可选) ==========================
% 如果不需要目录,注释掉下面两行
\tableofcontents
\newpage

% ========================== 正文开始 ==========================

\section{第一题}

\subsection{题目描述}
% 在这里写题目内容

\subsection{解答}

\solution

\answerbox{
% 在这里写你的解答
}

\section{第二题}

\subsection{题目描述}
% 在这里写题目内容

\subsection{解答}

\solution

\answerbox{
% 在这里写你的解答
}

% ========================== 示例内容 ==========================

\section{示例:数学公式}

行内公式:$E = mc^2$

独立公式:
\begin{equation}
\int_{-\infty}^{\infty} e^{-x^2} dx = \sqrt{\pi}
\label{eq:gaussian}
\end{equation}

多行对齐公式:
\begin{align}
\nabla \times \vec{E} &= -\frac{\partial \vec{B}}{\partial t} \\
\nabla \times \vec{B} &= \mu_0 \vec{J} + \mu_0 \epsilon_0 \frac{\partial \vec{E}}{\partial t}
\end{align}

矩阵:
\begin{equation}
\mat{A} = \begin{bmatrix}
a_{11} & a_{12} & a_{13} \\
a_{21} & a_{22} & a_{23} \\
a_{31} & a_{32} & a_{33}
\end{bmatrix}
\end{equation}

\section{示例:图片插入}

% 单张图片
\begin{figure}[H]
    \centering
    \includegraphics[width=0.6\textwidth]{example.jpg}
    \caption{示例图片}
    \label{fig:example}
\end{figure}

% 子图
\begin{figure}[H]
    \centering
    \begin{subfigure}{0.45\textwidth}
        \includegraphics[width=\textwidth]{image1.jpg}
        \caption{子图1}
    \end{subfigure}
    \hfill
    \begin{subfigure}{0.45\textwidth}
        \includegraphics[width=\textwidth]{image2.jpg}
        \caption{子图2}
    \end{subfigure}
    \caption{子图示例}
    \label{fig:subfigs}
\end{figure}

\section{示例:代码块}

Python代码:
\begin{lstlisting}[style=pythonstyle, caption=Python示例]
def fibonacci(n):
    """计算斐波那契数列第n项"""
    if n <= 1:
        return n
    return fibonacci(n-1) + fibonacci(n-2)

# 测试
for i in range(10):
    print(f"F({i}) = {fibonacci(i)}")
\end{lstlisting}

C++代码:
\begin{lstlisting}[style=cppstyle, caption=C++示例]
#include <iostream>
#include <vector>

int main() {
    std::vector<int> nums = {1, 2, 3, 4, 5};

    for (const auto& num : nums) {
        std::cout << num << " ";
    }
    std::cout << std::endl;

    return 0;
}
\end{lstlisting}

\section{示例:表格}

\begin{table}[H]
\centering
\caption{实验结果对比}
\begin{tabular}{|c|c|c|c|}
\hline
\textbf{方法} & \textbf{准确率} & \textbf{运行时间(s)} & \textbf{内存占用(MB)} \\
\hline
方法A & 95.2\% & 0.15 & 128 \\
\hline
方法B & 97.1\% & 0.23 & 156 \\
\hline
方法C & 96.8\% & 0.18 & 142 \\
\hline
\end{tabular}
\label{tab:results}
\end{table}

\section{示例:列表}

有序列表:
\begin{enumerate}
    \item 第一步:数据预处理
    \item 第二步:特征提取
    \item 第三步:模型训练
    \item 第四步:结果评估
\end{enumerate}

无序列表:
\begin{itemize}
    \item 优点1:计算效率高
    \item 优点2:内存占用少
    \item 优点3:易于实现
\end{itemize}

\section{示例:定理环境}

\begin{definition}
设 $f: \mathbb{R} \to \mathbb{R}$ 是一个函数,如果对于任意 $\epsilon > 0$,存在 $\delta > 0$,使得当 $|x - a| < \delta$ 时,有 $|f(x) - f(a)| < \epsilon$,则称函数 $f$ 在点 $a$ 处连续。
\end{definition}

\begin{theorem}
如果函数 $f$ 在闭区间 $[a,b]$ 上连续,则 $f$ 在 $[a,b]$ 上一致连续。
\end{theorem}

\begin{example}
函数 $f(x) = x^2$ 在任意有界闭区间上都是一致连续的。
\end{example}

% ========================== 参考文献 ==========================
\newpage
\section*{参考文献}
\addcontentsline{toc}{section}{参考文献}

\begin{thebibliography}{99}
\bibitem{ref1} 作者名. \textit{书名}. 出版社, 年份.
\bibitem{ref2} 作者名. 文章标题. \textit{期刊名}, 卷号(期号): 页码, 年份.
% 添加更多参考文献
\end{thebibliography}

% ========================== 附录(可选) ==========================
\newpage
\appendix
\section{附录:完整代码}
\label{app:code}

% 如果有代码文件,使用以下命令导入
% \lstinputlisting[caption=完整代码]{code_file.py}

\section{附录:详细计算过程}
\label{app:calc}

% 在这里放置详细的计算过程

\end{document}