\documentclass[12pt,a4paper]{article}

% ========================== 包导入 ==========================
\usepackage[UTF8]{ctex}                    % 中文支持
\usepackage[margin=2.2cm,top=2cm,bottom=2.5cm]{geometry}        % 页面设置
\usepackage{amsmath,amssymb,amsthm}        % 数学公式
\usepackage{graphicx}                      % 图片插入
\usepackage{float}                         % 图片位置控制
\usepackage{subcaption}                    % 子图支持
\usepackage{booktabs}                      % 表格美化
\usepackage{array}                         % 表格增强
\usepackage{enumitem}                      % 列表控制
\usepackage{xcolor}                        % 颜色支持
\usepackage{listings}                      % 代码块
\usepackage{fancyhdr}                      % 页眉页脚
\usepackage{hyperref}                      % 超链接
\usepackage{tikz}                          % 绘图
\usepackage{pgfplots}                      % 数据图表
\usepackage{algorithm}                     % 算法环境
\usepackage{algorithmic}                   % 算法伪代码
\usepackage{bm}                           % 粗体数学符号
\usepackage{url}                          % URL支持
\usepackage{tabularx}                     % 自适应表格
\usepackage{multirow}                     % 表格多行合并
\usepackage{footnote}                     % 脚注增强
\usepackage[most]{tcolorbox}              % 彩色文本框
\usepackage{titlesec}                     % 标题格式化

% ========================== 现代颜色定义 ==========================
\definecolor{pkured}{RGB}{153,28,40}      % 北大红
\definecolor{deepblue}{RGB}{0,73,144}     % 深蓝色
\definecolor{lightgray}{RGB}{248,249,250} % 浅灰色
\definecolor{codeback}{RGB}{247,247,247}  % 代码背景色
\definecolor{codeframe}{RGB}{221,221,221} % 代码边框色
\definecolor{accentblue}{RGB}{52,152,219} % 强调蓝色

% ========================== 可配置参数 ==========================
% 在这里修改课程信息
\newcommand{\coursename}{课程名称}             % 修改为你的课程名称
\newcommand{\hwnum}{1}                        % 修改为作业编号
\newcommand{\studentname}{姓名}                % 修改为你的姓名
\newcommand{\studentid}{学号}                  % 修改为你的学号
\newcommand{\department}{院系}                 % 修改为你的院系
\newcommand{\semester}{2025年秋季学期}         % 修改为当前学期

% ========================== 标题样式设置 ==========================
\titleformat{\section}
  {\Large\bfseries\color{pkured}}
  {\thesection}{1em}{}
  [\vspace{-0.5em}\rule{\textwidth}{0.5pt}]

\titleformat{\subsection}
  {\large\bfseries\color{deepblue}}
  {\thesubsection}{1em}{}

\titleformat{\subsubsection}
  {\normalsize\bfseries\color{accentblue}}
  {\thesubsubsection}{1em}{}

% ========================== 页眉页脚设置 ==========================
\pagestyle{fancy}
\fancyhf{}
\fancyhead[L]{\color{deepblue}\small\coursename\ 作业\hwnum}
\fancyhead[R]{\color{deepblue}\small 第 \thepage\ 页}
\fancyfoot[C]{\color{gray}\rule{0.8\textwidth}{0.5pt}}
\renewcommand{\headrulewidth}{1.2pt}
\renewcommand{\headrule}{\hbox to\headwidth{\color{pkured}\leaders\hrule height \headrulewidth\hfill}}
\renewcommand{\footrulewidth}{0pt}

% 行距设置
\linespread{1.3}

% 段落间距
\setlength{\parskip}{0.6em}
\setlength{\parindent}{0em}

% ========================== 现代代码设置 ==========================
% 定义更多现代化颜色
\definecolor{codeviolet}{RGB}{102,102,153}     % 紫色关键字
\definecolor{codegreen}{RGB}{0,128,0}          % 绿色注释
\definecolor{codeorange}{RGB}{255,102,0}       % 橙色字符串
\definecolor{codepurple}{RGB}{153,0,153}       % 紫色数字
\definecolor{codegray}{RGB}{128,128,128}       % 灰色行号
\definecolor{codecyan}{RGB}{0,153,153}         % 青色预处理
\definecolor{darkbackground}{RGB}{43,43,43}    % 深色背景
\definecolor{lightbackground}{RGB}{250,250,250} % 浅色背景

% VS Code 风格代码样式
\lstdefinestyle{vscode}{
    language=C++,
    basicstyle=\ttfamily\small\color{black},
    keywordstyle=\color{codeviolet}\bfseries,
    commentstyle=\color{codegreen}\itshape,
    stringstyle=\color{codeorange},
    numberstyle=\tiny\color{codegray},
    directivestyle=\color{codecyan}\bfseries,
    numbers=left,
    stepnumber=1,
    numbersep=15pt,
    backgroundcolor=\color{lightbackground},
    showspaces=false,
    showstringspaces=false,
    showtabs=false,
    frame=single,
    frameround=tttt,
    framesep=8pt,
    framerule=0.8pt,
    rulecolor=\color{accentblue!60},
    tabsize=4,
    captionpos=b,
    breaklines=true,
    breakatwhitespace=true,
    xleftmargin=25pt,
    xrightmargin=15pt,
    aboveskip=20pt,
    belowskip=20pt,
    columns=flexible,
    keepspaces=true,
    escapeinside={(*@}{@*)},
}

% GitHub 风格代码样式
\lstdefinestyle{github}{
    language=C++,
    basicstyle=\ttfamily\footnotesize\color{black},
    keywordstyle=\color{deepblue}\bfseries,
    commentstyle=\color{green!60!black}\itshape,
    stringstyle=\color{red!70!black},
    numberstyle=\scriptsize\color{gray!80},
    numbers=left,
    stepnumber=1,
    numbersep=12pt,
    backgroundcolor=\color{gray!3},
    showspaces=false,
    showstringspaces=false,
    showtabs=false,
    frame=tb,
    framesep=5pt,
    framerule=1pt,
    rulecolor=\color{gray!30},
    tabsize=4,
    captionpos=b,
    breaklines=true,
    breakatwhitespace=true,
    xleftmargin=20pt,
    xrightmargin=10pt,
    aboveskip=15pt,
    belowskip=15pt,
    columns=flexible,
    keepspaces=true,
}

% 极简风格代码样式
\lstdefinestyle{minimal}{
    language=C++,
    basicstyle=\ttfamily\small,
    keywordstyle=\color{pkured}\bfseries,
    commentstyle=\color{gray}\itshape,
    stringstyle=\color{accentblue},
    numberstyle=\tiny\color{gray!50},
    numbers=left,
    stepnumber=1,
    numbersep=10pt,
    backgroundcolor=\color{white},
    showspaces=false,
    showstringspaces=false,
    showtabs=false,
    frame=leftline,
    framesep=3pt,
    framerule=2pt,
    rulecolor=\color{pkured},
    tabsize=4,
    captionpos=b,
    breaklines=true,
    breakatwhitespace=true,
    xleftmargin=15pt,
    xrightmargin=5pt,
    aboveskip=12pt,
    belowskip=12pt,
}

% 设置默认样式为 VS Code 风格
\lstset{style=vscode}

% 提供样式切换命令
\newcommand{\setcodestyle}[1]{
    \lstset{style=#1}
}

% 多语言支持
\lstdefinelanguage{Python}{
    keywords={True, False, None, and, or, not, if, elif, else, for, while, in, def, class, return, yield, try, except, finally, with, as, import, from, lambda, pass, break, continue, global, nonlocal},
    keywordstyle=\color{codeviolet}\bfseries,
    ndkeywords={self, cls, __init__, __str__, __repr__, __len__},
    ndkeywordstyle=\color{codecyan}\bfseries,
    sensitive=true,
    comment=[l]{\#},
    string=[s]{"}{"},
    stringstyle=\color{codeorange},
    commentstyle=\color{codegreen}\itshape,
    morestring=[s]{"""}{"""},
    morestring=[s]{'''}{'''},
}

% 为不同语言提供特定样式
\lstdefinestyle{pythonstyle}{
    language=Python,
    basicstyle=\ttfamily\small\color{black},
    keywordstyle=\color{codeviolet}\bfseries,
    commentstyle=\color{codegreen}\itshape,
    stringstyle=\color{codeorange},
    numberstyle=\tiny\color{codegray},
    numbers=left,
    stepnumber=1,
    numbersep=15pt,
    backgroundcolor=\color{lightbackground},
    showspaces=false,
    showstringspaces=false,
    showtabs=false,
    frame=single,
    frameround=tttt,
    framesep=8pt,
    framerule=0.8pt,
    rulecolor=\color{accentblue!60},
    tabsize=4,
    captionpos=b,
    breaklines=true,
    breakatwhitespace=true,
    xleftmargin=25pt,
    xrightmargin=15pt,
    aboveskip=20pt,
    belowskip=20pt,
}

\lstdefinestyle{javastyle}{
    language=Java,
    basicstyle=\ttfamily\small\color{black},
    keywordstyle=\color{codeviolet}\bfseries,
    commentstyle=\color{codegreen}\itshape,
    stringstyle=\color{codeorange},
    numberstyle=\tiny\color{codegray},
    numbers=left,
    stepnumber=1,
    numbersep=15pt,
    backgroundcolor=\color{lightbackground},
    showspaces=false,
    showstringspaces=false,
    showtabs=false,
    frame=single,
    frameround=tttt,
    framesep=8pt,
    framerule=0.8pt,
    rulecolor=\color{accentblue!60},
    tabsize=4,
    captionpos=b,
    breaklines=true,
    breakatwhitespace=true,
    xleftmargin=25pt,
    xrightmargin=15pt,
    aboveskip=20pt,
    belowskip=20pt,
}

% ========================== 自定义框架环境 ==========================
\newtcolorbox{algorithmbox}[1]{
  colback=lightgray,
  colframe=deepblue,
  fonttitle=\bfseries\color{white},
  title=#1,
  sharp corners,
  boxrule=1pt,
  left=8pt,
  right=8pt,
  top=8pt,
  bottom=8pt
}

\newtcolorbox{keypoints}{
  colback=accentblue!5,
  colframe=accentblue,
  title=关键技术点,
  fonttitle=\bfseries\color{white},
  sharp corners,
  boxrule=1pt,
  left=8pt,
  right=8pt,
  top=8pt,
  bottom=8pt
}

% ========================== 数学设置 ==========================
\numberwithin{equation}{section}
\newtheorem{theorem}{定理}[section]
\newtheorem{lemma}[theorem]{引理}
\newtheorem{proposition}[theorem]{命题}
\newtheorem{corollary}[theorem]{推论}
\newtheorem{definition}{定义}[section]
\newtheorem{example}{例}[section]
\newtheorem{remark}{注}[section]

% ========================== 自定义命令 ==========================
% 向量表示
\newcommand{\vect}[1]{\boldsymbol{#1}}
% 矩阵表示
\newcommand{\mat}[1]{\boldsymbol{#1}}
% 现代解答框
\newenvironment{solution}
{\begin{tcolorbox}[colback=pkured!5,colframe=pkured,title=算法思路,fonttitle=\bfseries\color{white}]}
{\end{tcolorbox}}
% 重点标记
\newcommand{\important}[1]{\textcolor{pkured}{\textbf{#1}}}
% 注意事项
\newcommand{\note}[1]{\begin{tcolorbox}[colback=orange!10,colframe=orange!80,title=注意,fonttitle=\bfseries\color{white},sharp corners]#1\end{tcolorbox}}
% 提示
\newcommand{\hint}[1]{\begin{tcolorbox}[colback=accentblue!10,colframe=accentblue,title=提示,fonttitle=\bfseries\color{white},sharp corners]#1\end{tcolorbox}}

% ========================== 超链接设置 ==========================
\hypersetup{
    colorlinks=true,
    linkcolor=deepblue,
    filecolor=pkured,
    urlcolor=accentblue,
    citecolor=deepblue,
    pdftitle={\coursename 作业\hwnum},
    pdfauthor={\studentname},
    pdfsubject={\coursename},
    bookmarksnumbered=true,
    bookmarksopen=true
}

% ========================== 开始文档 ==========================
\begin{document}

% ========================== 现代标题页 ==========================
\begin{titlepage}
    \centering
    \vspace*{1cm}

    % 顶部装饰线
    {\color{pkured}\rule{\textwidth}{3pt}}
    \vspace{0.5cm}

    % 学校名称
    {\Huge\color{pkured}\textbf{北京大学}}\\[0.5cm]
    {\Large\color{deepblue}\coursename}\\[1cm]

    % 主标题区域
    \begin{tcolorbox}[
        colback=pkured!5,
        colframe=pkured,
        width=0.9\textwidth,
        arc=8pt,
        boxrule=2pt,
        title=作业报告,
        fonttitle=\Large\bfseries\color{white},
        center title
    ]
        \vspace{0.5cm}
        \begin{center}
            {\huge\color{pkured}\textbf{作业 \hwnum}}\\[0.3cm]
            {\Large\color{deepblue}\textbf{标题副标题}}
        \end{center}
        \vspace{0.5cm}
    \end{tcolorbox}

    \vspace{1.5cm}

    % 学生信息区域
    \begin{tcolorbox}[
        colback=lightgray,
        colframe=deepblue,
        width=0.7\textwidth,
        arc=5pt,
        boxrule=1pt
    ]
        \begin{center}
            \renewcommand{\arraystretch}{1.8}
            \begin{tabular}{rl}
                \color{deepblue}\textbf{姓\phantom{学}名:} & \large\studentname \\
                \color{deepblue}\textbf{学\phantom{学}号:} & \large\studentid \\
                \color{deepblue}\textbf{院\phantom{学}系:} & \large\department \\
                \color{deepblue}\textbf{课\phantom{学}程:} & \large\coursename \\
                \color{deepblue}\textbf{学\phantom{学}期:} & \large\semester
            \end{tabular}
        \end{center}
    \end{tcolorbox}

    \vfill

    % 底部装饰
    {\color{pkured}\rule{\textwidth}{2pt}}\\[0.3cm]
    {\large\color{deepblue} 提交日期:\today}\\[0.3cm]
    {\color{pkured}\rule{\textwidth}{1pt}}

\end{titlepage}

% ========================== 目录 ==========================
\tableofcontents
\newpage

% ========================== 正文开始 ==========================

\section{第一题}

\subsection{题目描述}
% 在这里写题目内容

\subsection{解答}

\begin{solution}
在这里写你的算法思路和解题步骤。使用这个彩色框来突出你的核心思想和方法。
\end{solution}

% VS Code 风格代码 (默认)
\begin{lstlisting}[style=vscode, caption=VS Code风格代码示例]
#include <iostream>
#include <vector>
#include <algorithm>

class DataProcessor {
public:
    void processData(const std::vector<int>& data) {
        // 使用现代C++特性进行数据处理
        std::for_each(data.begin(), data.end(), [](int value) {
            std::cout << "Processing: " << value << std::endl;
        });
    }
};

int main() {
    std::vector<int> data = {1, 2, 3, 4, 5};
    DataProcessor processor;
    processor.processData(data);
    return 0;
}
\end{lstlisting}

% GitHub 风格代码
\begin{lstlisting}[style=github, caption=GitHub风格代码示例]
def fibonacci(n):
    """计算第n个斐波那契数"""
    if n <= 1:
        return n
    return fibonacci(n-1) + fibonacci(n-2)

# 测试函数
for i in range(10):
    print(f"fibonacci({i}) = {fibonacci(i)}")
\end{lstlisting}

% 极简风格代码
\begin{lstlisting}[style=minimal, caption=极简风格代码示例]
public class QuickSort {
    public static void quickSort(int[] arr, int low, int high) {
        if (low < high) {
            int pi = partition(arr, low, high);
            quickSort(arr, low, pi - 1);
            quickSort(arr, pi + 1, high);
        }
    }

    private static int partition(int[] arr, int low, int high) {
        int pivot = arr[high];
        int i = low - 1;

        for (int j = low; j < high; j++) {
            if (arr[j] < pivot) {
                i++;
                swap(arr, i, j);
            }
        }
        swap(arr, i + 1, high);
        return i + 1;
    }
}
\end{lstlisting}

\begin{keypoints}
\begin{itemize}[leftmargin=15pt]
    \item \textbf{关键技术点1}:详细说明核心算法或技术要点
    \item \textbf{关键技术点2}:解释重要的实现细节
    \item \textbf{关键技术点3}:分析时间复杂度或空间复杂度
\end{itemize}
\end{keypoints}

\section{第二题}

\subsection{题目描述}
% 在这里写题目内容

\subsection{解答}

\begin{solution}
在这里写你的算法思路和解题步骤。
\end{solution}

% 插入图片示例
\begin{figure}[H]
    \centering
    \includegraphics[width=0.6\textwidth]{example.jpg}
    \caption{示例图片}
    \label{fig:example}
\end{figure}

\section{代码样式使用说明}

本模板提供三种现代化代码样式:

\begin{itemize}[leftmargin=20pt]
    \item \textbf{VS Code风格} (默认):\texttt{style=vscode} - 现代IDE风格,圆角边框,浅色背景
    \item \textbf{GitHub风格}:\texttt{style=github} - 简洁清爽,上下边框线
    \item \textbf{极简风格}:\texttt{style=minimal} - 左侧红色标线,纯净简约
\end{itemize}

你可以使用 \texttt{\textbackslash setcodestyle\{样式名\}} 来全局切换代码样式,或在单个代码块中指定 \texttt{style=样式名}。

支持多种编程语言:C++、Python、Java等,并提供语法高亮和关键字着色。

\section{示例:数学公式}

行内公式:$E = mc^2$

独立公式:
\begin{equation}
\int_{-\infty}^{\infty} e^{-x^2} dx = \sqrt{\pi}
\label{eq:gaussian}
\end{equation}

多行对齐公式:
\begin{align}
\nabla \times \vec{E} &= -\frac{\partial \vec{B}}{\partial t} \\
\nabla \times \vec{B} &= \mu_0 \vec{J} + \mu_0 \epsilon_0 \frac{\partial \vec{E}}{\partial t}
\end{align}

矩阵:
\begin{equation}
\mat{A} = \begin{bmatrix}
a_{11} & a_{12} & a_{13} \\
a_{21} & a_{22} & a_{23} \\
a_{31} & a_{32} & a_{33}
\end{bmatrix}
\end{equation}

\section{示例:表格}

\begin{table}[H]
\centering
\caption{实验结果对比}
\begin{tabular}{|c|c|c|c|}
\hline
\textbf{方法} & \textbf{准确率} & \textbf{运行时间(s)} & \textbf{内存占用(MB)} \\
\hline
方法A & 95.2\% & 0.15 & 128 \\
\hline
方法B & 97.1\% & 0.23 & 156 \\
\hline
方法C & 96.8\% & 0.18 & 142 \\
\hline
\end{tabular}
\label{tab:results}
\end{table}

\section{示例:特殊框架}

\note{这是一个注意事项的示例。用来提醒重要的注意点或者警告信息。}

\hint{这是一个提示框的示例。用来给出有用的提示或者建议。}

\begin{algorithmbox}{算法描述}
这是一个算法描述框的示例。可以用来详细描述算法的步骤和流程。
\end{algorithmbox}

\section{总结}

本次作业通过实践加深了对相关知识点的理解。主要收获包括:

\begin{enumerate}[leftmargin=20pt]
    \item 掌握了核心算法的实现原理
    \item 学会了代码优化和性能分析
    \item 提高了问题分析和解决能力
\end{enumerate}

% ========================== 参考文献 ==========================
\section*{参考文献}

\begin{enumerate}
    \item 作者名. \textit{书名}. 出版社, 年份.
    \item 作者名. 文章标题. \textit{期刊名}, 卷号(期号): 页码, 年份.
\end{enumerate}

\end{document}